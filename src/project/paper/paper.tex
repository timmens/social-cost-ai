\documentclass[a4paper, 12pt]{article}

\usepackage[T1]{fontenc}
\usepackage[utf8]{inputenc}
\usepackage[left=1in, right=1in, top=1in, bottom=1in]{geometry}
\usepackage[affil-it]{authblk}

\usepackage{amsmath}
\usepackage{amssymb}
\usepackage{hyperref}
\usepackage{mathpazo}
\usepackage{tgpagella}
\usepackage{xcolor}

\counterwithin{figure}{section}
\linespread{1.25}

%%%%%%%%%%%%%%%%%%%%%%%%%%%%%%%%%%%%%%%%%%%%%%%%%%
% Bibliography settings
%%%%%%%%%%%%%%%%%%%%%%%%%%%%%%%%%%%%%%%%%%%%%%%%%%
\usepackage[backend=biber, style=ieee, autocite=inline, doi=false, url=true]{biblatex}
\DeclareLanguageMapping{english}{english-apa}
\setcounter{biburlnumpenalty}{85}  % for urls that are too long
\addbibresource{bibliography.bib}
\renewcommand*{\bibfont}{\footnotesize}
\AtEveryBibitem{%
  \clearfield{volume}%
  \clearfield{number}%
  \clearfield{pages}%
}

\newcommand{\mycite}[1]{\citeauthor*{#1} (\citeyear{#1}, \cite{#1})}

%%%%%%%%%%%%%%%%%%%%%%%%%%%%%%%%%%%%%%%%%%%%%%%%%%
% Colors
%%%%%%%%%%%%%%%%%%%%%%%%%%%%%%%%%%%%%%%%%%%%%%%%%%
\definecolor{bonnblue}{RGB}{4, 83, 156}  % hex: #04539C
\definecolor{bonngrey}{RGB}{148, 147, 132}  % hex: #8A9384
\definecolor{bonnyellow}{RGB}{251, 187, 6}  % hex: #FBBB06
\definecolor{bonnred}{RGB}{178, 34, 34}  % hex: #B22222

%%%%%%%%%%%%%%%%%%%%%%%%%%%%%%%%%%%%%%%%%%%%%%%%%%
% Auxiliary commands
%%%%%%%%%%%%%%%%%%%%%%%%%%%%%%%%%%%%%%%%%%%%%%%%%%

\newcommand{\blue}[1]{\textcolor{bonnblue}{#1}}
\newcommand{\yellow}[1]{\textcolor{bonnyellow}{#1}}
\newcommand{\grey}[1]{\textcolor{bonngrey}{#1}}
\newcommand{\red}[1]{\textcolor{bonnred}{#1}}

%% Hyperlink stuff
\renewcommand\thefootnote{\blue{\arabic{footnote}}}

\urlstyle{tt}
\hypersetup{
    colorlinks=true,
    urlcolor=bonnblue,
    citecolor=bonnblue,
    allcolors=bonnblue,
}


%%%%%%%%%%%%%%%%%%%%%%%%%%%%%%%%%%%%%%%%%%%%%%%%%%
% Title page
%%%%%%%%%%%%%%%%%%%%%%%%%%%%%%%%%%%%%%%%%%%%%%%%%%
\title{\textbf{The Social Costs of AI}\\
    \Large \red{Working Paper}
}
\date{Version: \today}
\author[1]{Aimee van Wynsberghe}
\author[2]{Tim Mensinger}
\author[3]{Stefan H{\"o}se}
\affil[1]{Institute of Science and Ethics, University of Bonn}
\affil[2]{Institute of Finance and Statistics, University of Bonn}
\affil[3]{DRZE, University of Bonn}



%%% title page
\title{\textbf{The Social Costs of AI}\\
    \Large Literature Review
}
\date{}
\author{Tim Mensinger%
  \thanks{\texttt{tmensinger[at]uni-bonn.de}}
}
\affil{Bonn Graduate School of Economics}


\begin{document}

\onehalfspacing


\maketitle

%%% preamble (for senior researcher)
\renewcommand{\abstractname}{Preamble}
\begin{abstract}
    \noindent This document is intended as a reference guide to the existing literature
    and available online tools. Connections between your research question and the
    existing literature are highlighted and, if necessary, formalized. The abstract
    below is designed to clarify that we are analyzing the same research question.

    All materials needed to reproduce this project can be found online in a
    \emph{private} repository\footnote{Online repository:
    \boldurl{https://github.com/timmens/social-cost-ai}}.
\end{abstract}

%%% abstract
\renewcommand{\abstractname}{Abstract}
\begin{abstract}
    \noindent Modern AI research and its application in industry have developed rapidly
    over the last decades. We claim that the AI community has (implicitly) defined their
    success using a metric which focuses solely on an improvement of model accuracy. In
    this essay, we further argue that by ignoring other social criteria (e.g.
    environmental or economic) the community can drift towards a socially suboptimal
    equilibrium.
\end{abstract}
\thispagestyle{empty}

%%% contents
\newpage

%%% toc
% {
%     \hypersetup{linkcolor=bonnblue}
%     \tableofcontents
% }

%%% introduction

\section{Introduction}

There is a small and recent literature tackling questions concerning the social costs of
AI. In the following sections I will review the papers I found to be most similar to
what you talked about. A list of further potentially interesting papers is appended to
the bibliography.


\section{Literature Review}

\subsection{Quantifying the Carbon Emissions of Machine Learning}

\subsection{\textit{Green AI}}
\centerline{\mycite{Schwartz_2020}}
\vspace{1em}

\paragraph{Summary:} AI field is focusing too much on \emph{model accuracy}, which leads
to rapidly increasing computational costs, especially since returns of accuracy are
diminishing with respect to computational costs. This then leads to crowding-out of
researchers or companies with little funds, and high economic and environmental costs
($\approx$ computational costs). They term this development \emph{Red AI} and present
their idea of \emph{Green AI}, which takes into account other metrics that are supposed
to account for social costs (economic or environmental). As a specific metric to judge
how good a new model is, they propose FPO (floating point operations), because these are
hardware and location independent. They acknowladge the work of \mycite{Lacoste_2019}
and \mycite{Henderson_2020}, but believe that their direct measure of $CO_2$ equivalents
is inferior, since it looses the connection to the model because much of the variability
of the measure stems from local energy efficiency and hardware choice.

\subsection{\textit{Energy and Policy Considerations for Modern Deep Learning Research}}
\centerline{\mycite{Strubell_2020}}
\vspace{1em}

\paragraph{Summary:} The paper presents a strategy on how to (point) estimate $CO_2$
equivalents and economic costs of AI model training. They then apply this strategy to a
variety of NLP models and report the costs of training and the costs of development. The
costs of development takes into account that there are certain hyperparameters that need
to be optimized outside the training procedure. They find that a single training process
of a standard NLP model ($BERT_{base}$) creates approximately as much $CO_2$ equivalents
as a flight from NY to SF. In their case-study they consider a simpler model for which
they find that the full development (repeated training of 5000 models instead of 1
model) increases the costs by a factor of 2000. The factor is not one-to-one with the
number of repetitions since there was a possibility for jobs to crash early. They
conclude that in order to compare training costs across models independent of hardware
one should use FPO, as mentioned in \mycite{Schwartz_2020}.

\paragraph{Formula:} Let $p_c$, $p_g$, $p_r$ denote the average power consumption in
Watts of the CPU, GPU and DRAM (memory), respectively. Let $g$ denote the number of GPUs
used for training. Let 1.58 denote the Power Usage Effectiveness (PUE) coefficient. The
total power consumption at a given instance is then given by
\begin{align}
    p_t = \frac{1.58 (p_c + p_r + g p_g)}{1000} \,,
\end{align}
where a division by 1000 transforms Watts to KiloWatts. To compute $CO_2$ equivalents
one can use
\begin{align}
    CO_2e = 0.954 p_t \,.
\end{align}


\nocite{*}
\printbibliography[heading=bibintoc, title={Bibliography}]

\end{document}
