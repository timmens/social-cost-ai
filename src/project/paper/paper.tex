\documentclass[a4paper, 12pt]{article}

\usepackage{sourcesanspro}
\usepackage[T1]{fontenc}
\usepackage[utf8]{inputenc}
\usepackage[document]{ragged2e}
\usepackage{hyperref}
\usepackage{fontawesome5}
\usepackage[tracking=true]{microtype}
\usepackage[scaled=0.90]{helvet}
\usepackage{caption}
\usepackage[scaled=0.85]{beramono}
\usepackage{tgpagella}
\usepackage{mathpazo}
\usepackage{mathtools}
\usepackage{amsmath}
\usepackage{amssymb}
\usepackage{amsthm}
\usepackage{scalerel}
\usepackage{bm}
\usepackage{bbm}
\usepackage{xcolor}
\usepackage{enumerate}
\usepackage[percent]{overpic}
\usepackage{soul}
\usepackage{tikz}
\usepackage{listings}
\usepackage{lstbayes}
\usepackage{booktabs}
\usepackage{bold-extra}
\usepackage{textpos}
\usepackage{multirow}
\usepackage{longtable}
\usepackage{tabularx}
\usepackage[flushleft]{threeparttable}
\usepackage{varwidth}

\usepackage{xpatch}
\usepackage[plain]{algorithm}
\usepackage{algorithmicx}
\usepackage[noend]{algpseudocode}

\usepackage{graphicx}
\usepackage{fancybox}


\usepackage[skins,theorems]{tcolorbox}
\tcbset{highlight math style={enhanced,colframe=red,colback=white,arc=0pt,boxrule=1pt}}

%%%%%%%%%%%%%%%%%%%%%%%%%%%%%%%%%%%%%%%%%%%%%%%%%%
% Bibliography settings
%%%%%%%%%%%%%%%%%%%%%%%%%%%%%%%%%%%%%%%%%%%%%%%%%%
\usepackage[backend=biber, style=apa, autocite=inline, doi=false, url=false]{biblatex}
\DeclareLanguageMapping{english}{english-apa}
\setcounter{biburlnumpenalty}{85}  % for urls that are too long
\addbibresource{bibliography.bib}
\setbeamertemplate{bibliography item}{}% Remove reference icon.
\setbeamercolor{bibliography entry author}{fg=bonnblue}
\setbeamercolor{bibliography entry note}{fg=bonnblue}
\renewcommand*{\bibfont}{\footnotesize}
\setlength\bibitemsep{\itemsep}
\AtEveryBibitem{%
  \clearfield{volume}%
  \clearfield{number}%
  \clearfield{pages}%
}
\newcommand\mycite[1]{\textcolor{bonnblue}{\citeauthor{#1}} (\textcolor{bonnblue}{\citeyear{#1}})}


%%%%%%%%%%%%%%%%%%%%%%%%%%%%%%%%%%%%%%%%%%%%%%%%%%
% Algorithm
%%%%%%%%%%%%%%%%%%%%%%%%%%%%%%%%%%%%%%%%%%%%%%%%%%
\makeatletter
\xpatchcmd{\algorithmic}{\itemsep\z@}{\itemsep=1ex plus1pt}{}{}
\makeatother



%%%%%%%%%%%%%%%%%%%%%%%%%%%%%%%%%%%%%%%%%%%%%%%%%%
% Math definitions
%%%%%%%%%%%%%%%%%%%%%%%%%%%%%%%%%%%%%%%%%%%%%%%%%%
\newcommand\indicator[1]{\scaleobj{1.4}{\mathbbm{1}} \! \left\{#1\right\}}
\newcommand\expectation[1]{\mathbb{E}\left[#1\right]}

\DeclareMathOperator*{\argmax}{argmax}


%%%%%%%%%%%%%%%%%%%%%%%%%%%%%%%%%%%%%%%%%%%%%%%%%%
% Colors
%%%%%%%%%%%%%%%%%%%%%%%%%%%%%%%%%%%%%%%%%%%%%%%%%%
\definecolor{bonnblue}{RGB}{4, 83, 156}  % hex: #04539C
\definecolor{bonngrey}{RGB}{148, 147, 132}  % hex: #8A9384
\definecolor{bonnyellow}{RGB}{251, 187, 6}  % hex: #FBBB06
\definecolor{bonngreen}{RGB}{0, 128, 0}  % hex: #FBBB06
\definecolor{bonnred}{RGB}{153, 0, 0}  % hex: #FBBB06
\definecolor{red}{RGB}{65, 16, 16}  % hex: #411010

%%%%%%%%%%%%%%%%%%%%%%%%%%%%%%%%%%%%%%%%%%%%%%%%%%
% Auxiliary files
%%%%%%%%%%%%%%%%%%%%%%%%%%%%%%%%%%%%%%%%%%%%%%%%%%
\newcommand{\email}[1]{\def\@email{\texttt{\MakeLowercase{\textls[10]{#1}}}}}

\setbeamerfont{title}{series=\bfseries} % family=\sourcesanspro
\setbeamerfont{subtitle}{series=\scshape, size*={8}{15}} % family=\sourcesanspro
\setbeamerfont{author}{family=\scshape}

\setbeamercolor{title}{fg=black}
\setbeamercolor{subtitle}{fg=bonngrey}
\setbeamercolor{author}{fg=bonnblue}
\setbeamercolor{date}{fg=bonnyellow}
\setbeamercolor{institute}{fg=bonngrey}
\setbeamercolor{email}{fg=bonnyellow}

\defbeamertemplate*{title page}{customized}[1][]
{
    \usebeamerfont{title}\usebeamercolor[fg]{title}\textls[100]{\MakeUppercase{\textbf{\inserttitle}}}\par
    \usebeamerfont{subtitle}\usebeamercolor[fg]{subtitle}\textls[100]{\textsc{\insertsubtitle}}\par\par
    {\usebeamercolor[fg]{date}\usebeamerfont{date}\insertdate\par}
    \vfill
    \usebeamerfont{author}\usebeamercolor[fg]{author}\textls[100]{\textsc{\insertauthor}}\par
    \usebeamerfont{institute}{\usebeamercolor[fg]{institute}\textls[100]{\textsc{\insertinstitute}}}\par
    \bigskip
}



%% Geometry

\geometry{left=1cm, right=1.5cm}


%%%%%%%%%%%%%%%%%%%%%%%%%%%%%%%%%%%%%%%%%%%%%%%%%%
% Auxiliary commands
%%%%%%%%%%%%%%%%%%%%%%%%%%%%%%%%%%%%%%%%%%%%%%%%%%
% \graphicspath{files/} % graphics path
\newcommand{\labelitem}{\tikz\draw[black,fill=bonnblue] (0,0) circle (.35ex); \,\,}

\newcommand{\blue}[1]{\textcolor{bonnblue}{#1}}
\newcommand{\yellow}[1]{\textcolor{bonnyellow}{#1}}
\newcommand{\grey}[1]{\textcolor{bonngrey}{#1}}
\newcommand{\red}[1]{\textcolor{red}{#1}}
\newcommand{\green}[1]{\textcolor{bonngreen}{#1}}
\def\ruleline#1{\hbox to \hsize{\hrulefill\quad\lower.3em\hbox{#1}\quad\hrulefill}}


\newcommand{\mycheck}{\textcolor{bonngreen}{\faIcon{check}}}
\newcommand{\mycross}{\textcolor{bonnred}{\faIcon{times}}}


%%%%%%%%%%%%%%%%%%%%%%%%%%%%%%%%%%%%%%%%%%%%%%%%%%
% Beamer settings
%%%%%%%%%%%%%%%%%%%%%%%%%%%%%%%%%%%%%%%%%%%%%%%%%%
\setbeamercolor{itemize item}{fg=black}
\setbeamercolor{itemize subitem}{fg=black}
\setbeamercolor{itemize subsubitem}{fg=black}
\setbeamercolor{section head}{fg=cardinal} % currently unused.
\setbeamercolor{footline}{fg=bonnblue}
\setbeamercolor{section in toc}{fg=black}

\setbeamertemplate{itemize item}[circle]
\setbeamertemplate{itemize subitem}{{\textendash}}
\setbeamertemplate{itemize subsubitem}[triangle]
\setbeamertemplate{section in toc}{{\inserttocsectionnumber.}~\inserttocsection}

\setbeamerfont{section in toc}{series=\scshape, size=\Large}
\setbeamerfont{frametitle}{series=\scshape} % \itshape
\setbeamercolor{frametitle}{fg=bonnblue}

\setbeamertemplate{frametitle}
{
    \vspace*{0.4cm}
    \insertframetitle \\ \footnotesize \textcolor{bonngrey}{\insertframesubtitle}
}

\AtBeginSection[]{
  \setbeamertemplate{footline}[frame number]{}
  \setbeamertemplate{navigation symbols}{}
  \setbeamertemplate{footline}{}
  \begin{frame}
  \vfill
  \centering
  \begin{beamercolorbox}[sep=8pt,center]{title}
      {\usebeamerfont{title}\usebeamercolor[fg]{title}{\ruleline{\textsc{\insertsectionhead}}}}\par%
  \end{beamercolorbox}
  \vfill
  \end{frame}
  \addtobeamertemplate{navigation symbols}{}{%
      \usebeamerfont{footline}%
      \usebeamercolor[fg]{author}%
      \hspace{1em}%
      \raisebox{2em}[0pt][0pt]{%
      \normalsize \insertframenumber \kern 0.25em
      }
  }
}

\AtBeginSubsection[]{
  \setbeamertemplate{footline}[frame number]{}
  \setbeamertemplate{navigation symbols}{}
  \setbeamertemplate{footline}{}
  \begin{frame}
  \vfill
  \centering
  \begin{beamercolorbox}[sep=8pt,center]{title}
      {\usebeamerfont{title}\usebeamercolor[fg]{title}{\textsc{{\normalsize
      \grey{\insertsectionhead}}\\[1em] \insertsubsectionhead}}}\par%
  \end{beamercolorbox}
  \vfill
  \end{frame}
  \addtobeamertemplate{navigation symbols}{}{%
      \usebeamerfont{footline}%
      \usebeamercolor[fg]{author}%
      \hspace{1em}%
      \raisebox{2em}[0pt][0pt]{%
      \normalsize \insertframenumber \kern 0.25em
      }
  }
}



%%% title page
\title{\textbf{The Social Costs of AI}\\
    \Large Literature Review
}
\date{}
\author{Tim Mensinger%
  \thanks{\texttt{tmensinger[at]uni-bonn.de}}
}
\affil{Bonn Graduate School of Economics}


\begin{document}

\onehalfspacing


\maketitle

%%% preamble (for senior researcher)
\renewcommand{\abstractname}{Preamble}
\begin{abstract}
    \noindent This document is intended as a reference guide to the existing literature
    and available online tools. Connections between your research question and the
    existing literature are highlighted and, if necessary, formalized. The abstract
    below is designed to clarify that we are analyzing the same research question.

    All materials needed to reproduce this project can be found online in a
    \emph{private} repository.\footnote{Online repository:
    \boldurl{https://github.com/timmens/social-cost-ai}. To access the repository one
    needs a GitHub account plus an invitation.}
\end{abstract}

%%% abstract
\renewcommand{\abstractname}{Abstract}
\begin{abstract}
    \noindent Modern AI research and its application in industry have developed rapidly
    over the last decades. We claim that the AI community has (implicitly) defined their
    success using a metric which focuses solely on an improvement of model accuracy. In
    this essay, we further argue that by ignoring other social criteria (e.g.
    environmental or economic) the community can drift towards a socially suboptimal
    equilibrium. However, in order to accurately model the trade-offs between social
    criteria and model improvement one needs to establish a methodology which can be
    used to measure the (environmental) impact of running machine learning models.
\end{abstract}
\thispagestyle{empty}

%%% contents
\newpage

%%% toc
% {
%     \hypersetup{linkcolor=bonnblue}
%     \tableofcontents
% }

%%% introduction

\section{Literature Review}

There is a small and recent literature tackling questions concerning the social costs of
AI. In the following sections I will review the papers I found to be most similar to
what you talked about. The ordering is roughly chronological. If you want to read one
paper only, read \mycite{Henderson_2020}. All of the papers below also discuss
mitigation strategies which I ignore in my summaries.


\subsection{Quantifying the Carbon Emissions of Machine Learning}
\centerline{\mycite{Lacoste_2019}}
\vspace{1em}

\paragraph{Summary:} Explains that the AI community ---research and industry--- is using
ever-growing computational resources, which can be translated directly to growing energy
needs. The paper then argues that in order to start a founded discussion on the
tradeoffs occuring because of the (potentially significant) climate impact of machine
learning, one needs methodology that approximates the impact of a model (training,
deploying). They then present their own implementation of such an approximation model.
The model, or "ML $CO_2$ impact calculator" as they call it, is hosted
online.\footnote{\boldurl{https://mlco2.github.io/impact/}} The calculator gives a rough
estimate of the carbon emission produced through energy consumption. They do this using
information on the hardware type, hours for which the model ran, and carbon efficieny of
the local power grid. Certain cloud computing infrastructures are also allowed (Google,
AWS, Azure). There are many underlying assumptions needed in order for this estimate to
be accurate. Some are listed and critized in \mycite{Henderson_2020}. In my view, this
calculator can be used to get a first estimate. Because of their underlying assumptions
the estimate will tend to be positively biased. Hence, if one uses conservative values
for the energy efficiency, the estimate should provide an upper bound of the true value.

\paragraph{Formula:} The formula is not described in the paper and I did not manage to
find the time to extract the (mathematical) formula from their code.

\paragraph{Computational implementation:} I have not looked at their computational
implementation of the calculator in detail. Hence I cannot judge whether the
implementation is trustworthy.


\subsection{Energy and Policy Considerations for Modern Deep Learning Research}
\label{seq:Strubell_2020}
\centerline{\mycite{Strubell_2020}}
\vspace{1em}

\paragraph{Summary:} The paper presents a strategy on how to (point) estimate $CO_2$
equivalents and economic costs of AI model training. They then apply this strategy to a
variety of NLP models and report the costs of training and the costs of development. The
costs of development takes into account that there are certain hyperparameters that need
to be optimized outside the training procedure. They find that a single training process
of a standard NLP model ($BERT_{base}$) creates approximately as much $CO_2$ equivalents
as a flight from NY to SF. In their case-study they consider a simpler model for which
they find that the full development (repeated training of 5000 models instead of 1
model) increases the costs by a factor of 2000. The factor is not one-to-one with the
number of repetitions since there was a possibility for jobs to crash early. They
conclude that in order to compare training costs across models independent of hardware
one should use FPO, as mentioned in \mycite{Schwartz_2020}.

\paragraph{Formula:} Let $p_c$, $p_g$, $p_r$ denote the average power consumption in
Watts of the CPU, GPU and DRAM (memory), respectively. Let $g$ denote the number of GPUs
used for training. Let 1.58 denote the Power Usage Effectiveness (PUE) coefficient. The
total power consumption at a given instance is then given by
\begin{align}
    p_t = \frac{1.58 (p_c + p_r + g p_g)}{1000} \,,
\end{align}
where a division by 1000 transforms Watts to KiloWatts. To compute $CO_2$ equivalents
one can use
\begin{align}
    CO_2e = 0.954 p_t \,.
\end{align}


\subsection{Towards the Systematic Reporting of the Energy and Carbon Footprints of
Machine learning}
\label{seq:Henderson_2020}
\centerline{\mycite{Henderson_2020}}
\vspace{1em}

\paragraph{Summary:} The paper lists similar arguments as above for why a methodology to
accurately measure the carbon impact of machine learning models is needed; see
\mycite{Strubell_2020}, \mycite{Schwartz_2020} and \mycite{Lacoste_2019}. The main
contributions are a (legal) policy viewpoint on the matter as well as a very advanced
software implementation designed to measure the carbon impact of machine learning models
in real time. In comparison to the aforementioned papers, their approach is to model the
energy consumption in as much detail as possible.

The micro-based idea of modelling each component of the system in great detail is only
favorable if we can trust the data on the micro-level. Otherwise aggregation can result
in the accumulation of errors. Especially problematic is that we loose the information
on the direction of the bias, whereas above we knew that we were over-estimating.

\paragraph{Formula:} The formula is similar to the one used in \mycite{Strubell_2020};
see section \ref{seq:Strubell_2020}. Let $PUE$ denote the power usage effectiveness
coefficient. Now consider the set of all (computer) processes that are spawned during a
model run and denote it by $\mathcal{P}$. In comparison to section
\ref{seq:Strubell_2020} here $e$ denotes energy, while $p$ denotes the percentage each
resource uses by the attributable process. As an example, given a process $\rho$, the
CPU uses energy $e_{CPU}$ and $p_{CPU} = p_{CPU}(\rho)$ is the utilization of the CPU
from process $\rho$. The total energy consumed by process $\rho$ from the CPU is then
given by $p_{CPU} e_{CPU}$, where we drop the dependency on $\rho$ for the sake of
clarity. The overall energy consumption is then given by a double sum over the multiple
components and processes, multiplied with eh $PUE$ coefficient:
\begin{align}
    e_{total} = PUE \sum_{\rho \in \mathcal{P}} (p_{dram} e_{dram} + p_{cpu} e_{cpu} +
    p_{gpu} e_{gpu})
\end{align}

\paragraph{Computational Implementation:} The computational implementation of their
calculator seems to be the most advanced. However, it is still in development status and
buggy. Unfortunately it seems like development stopped 7 months ago (January, 2022). If
development does not continue, or is picked up by someone else, I cannot recommend using
this software without applying custom fixes.


\subsection{Green AI}
\centerline{\mycite{Schwartz_2020}}
\vspace{1em}

\paragraph{Summary:} AI field is focusing too much on \emph{model accuracy}, which leads
to rapidly increasing computational costs, especially since returns of accuracy are
diminishing with respect to computational costs. This then leads to crowding-out of
researchers or companies with little funds, and high economic and environmental costs
($\approx$ computational costs). They term this development \emph{Red AI} and present
their idea of \emph{Green AI}, which takes into account other metrics that are supposed
to account for social costs (economic or environmental). As a specific metric to judge
how good a new model is, they propose FPO (floating point operations), because these are
hardware and location independent. They acknowladge the work of \mycite{Lacoste_2019}
and \mycite{Henderson_2020}, but believe that their direct measure of $CO_2$ equivalents
is inferior, since it looses the connection to the model because much of the variability
of the measure stems from local energy efficiency and hardware choice.


\nocite{*}
\printbibliography[heading=bibintoc, title={Bibliography}]

\end{document}
