
\section{Literature Review}

There is a small, recent literature that addresses questions about the social costs of
AI. In the following sections, I will review the works that I think are most similar to
what you talked about. The order is roughly chronological. If you want to read only one
paper, read \mycite{Henderson_2020}. All of the following papers also discuss mitigation
strategies, which I ignore in my summaries.


\subsection{Quantifying the Carbon Emissions of Machine Learning}
\centerline{\mycite{Lacoste_2019}}
\vspace{1em}

\paragraph{Summary:} Explains that the AI community ---research and industry--- is
consuming more and more computational resources, which translates directly into
increasing energy demand. The paper then argues that an informed discussion of the
tradeoffs that arise due to the (potentially significant) climate impact of machine
learning requires a methodology that approximates the impact of a model (training,
deployment). They then present their own implementation of such an approximation model.
The model, or "ML $CO_2$ impact calculator" as they call it, is hosted
online.\footnote{\url{https://mlco2.github.io/impact/}} The calculator gives a rough
estimate of the carbon emissions caused by energy use.  It does this using information
about the type of hardware, the hours the model was running, and the CO2 efficiency of
the local power grid. Certain cloud computing infrastructures are also allowed (Google,
AWS, Azure). For this estimate to be accurate, many assumptions must be made. Some of
these are discussed in \mycite{Henderson_2020}. In my view, this calculator can be
used to obtain an initial estimate. Because of the underlying assumptions, the estimate
will tend to be positively biased. Thus, if one uses conservative values for energy
efficiency, the estimate should represent an upper bound of the true value.

\paragraph{Formula:} The formula is not described in the paper, and I have not managed
to find the time to extract the (mathematical) formula from their code.

\paragraph{Computational implementation:} I have not looked at the computational
implementation of the calculator in detail. Therefore, I cannot judge whether the
implementation is trustworthy.


\subsection{Energy and Policy Considerations for Modern Deep Learning Research}
\label{seq:Strubell_2020}
\centerline{\mycite{Strubell_2020}}
\vspace{1em}

\paragraph{Summary:} The paper presents a strategy for (pointwise) estimating the
$CO_2$ equivalents and economic costs of training AI models. They then apply this
strategy to a set of NLP models and report the costs of training and the development
costs. The development costs take into account that there are certain hyperparameters
that need to be optimized outside of the training process. They find that a single
training process of a standard NLP model ($BERT_{base}$) generates about as much $CO_2$
equivalents as a flight from NY to SF. In their case study, they consider a simpler
model, for which they find that full development (repeated training of 5000 models
instead of 1 model) increases the cost by a factor of 2000. The factor is not one-to-one
with the number of repetitions, since there was a possibility that jobs were canceled
early. They conclude that to compare training costs for different models, independent of
hardware and local power grid specifications, one should use FPO, as mentioned in
\mycite{Schwartz_2020}.

\paragraph{Formula:} Let $p_c$, $p_g$, $p_r$ denote the average power consumption in
Watts of the CPU, GPU and DRAM (memory), respectively. Let $g$ denote the number of GPUs
used for training. Let 1.58 denote the PUE (Power Usage Effectiveness) coefficient. The
total power consumption at a given instance is then given by
\begin{align}
    p_t = \frac{1.58 (p_c + p_r + g p_g)}{1000} \,,
\end{align}
where a dividing by 1000 converts Watts to KiloWatts. To calculate the $CO_2$
equivalents, one can use
\begin{align}
    CO_2e = 0.954 p_t \,.
\end{align}


\subsection{Towards the Systematic Reporting of the Energy and Carbon Footprints of
Machine learning}
\label{seq:Henderson_2020}
\centerline{\mycite{Henderson_2020}}
\vspace{1em}

\paragraph{Summary:} The paper makes similar arguments as above as to why methods to
accurately measure the carbon impact of machine learning models are needed; see
\mycite{Strubell_2020}, \mycite{Schwartz_2020}, and \mycite{Lacoste_2019}. The main
contributions are a (legal) policy viewpoint on this topics and a very advanced software
implementation to measure the carbon impact of machine learning models in real time.
Compared to the aforementioned work, their approach is to model energy consumption in as
much detail as possible.

The micro-based idea of modeling each component of the system in minute detail is only
beneficial if we can trust the micro-level data. Otherwise, aggregation can lead to an
accumulation of errors. Particularly problematic is that we lose the information about
the direction of the bias, while we knew above that we overestimated the true value.

\paragraph{Formula:} The formula is similar to that used in \mycite{Strubell_2020}; see
Section \ref{seq:Strubell_2020}. $PUE$ stands for the power usage effectiveness
coefficient. Consider the set of all (computer) processes spawned during a model run and
denote it by $\mathcal{P}$. Compared to Section \ref{seq:Strubell_2020}, here $e$
denotes energy, while $p$ denotes the percentage of each resource used by the
attributable process. For example, for a process $\rho$, the CPU consumes energy
$e_{CPU}$, and $p_{CPU} = p_{CPU}(\rho)$ is the utilization of the CPU by process
$\rho$. The total energy consumption of process $\rho$ by the CPU is then given by
$p_{CPU} e_{CPU}$, where we omit the dependence on $\rho$ for clarity. The total energy
consumption is then defined by a double sum over the different components and processes,
multiplied with the $PUE$ coefficient:
\begin{align}
    e_{total} = PUE \sum_{\rho \in \mathcal{P}} (p_{dram} e_{dram} + p_{cpu} e_{cpu} +
    p_{gpu} e_{gpu})
\end{align}

\paragraph{Computational Implementation:} The computational implementation of their
calculator seems to be the most advanced. However, it is still in development stage and
buggy. Unfortunately, development appears to have ceased in July 2021. Unless
development continues or is taken over by someone else, I cannot recommend using this
software without making your own corrections.


\subsection{Green AI}
\centerline{\mycite{Schwartz_2020}}
\vspace{1em}

\paragraph{Summary:} The paper argues that the field of AI is overly focused on
\emph{model accuracy}, leading to rapidly increasing computational costs, especially as
the returns to accuracy are diminish relative to computational costs. This then leads to
crowding out of researchers or companies with low resources as well as high economic and
environmental costs ($\approx$ computational costs). They refer to this trend as
\emph{Red AI} and introduce their idea of \emph{Green AI}, which considers other metrics
that should take into account social costs (economic or environmental). The propose FPO
(Floating Point Operations) as a specific metric to assess the quality of a new model,
since it is hardware and locations independent. They acknowledge the work of
\mycite{Lacoste_2019} and \mycite{Henderson_2020}, but believe that their direct measure
of $CO_2$ equivalents is inferior because it loses the link to the model, as much of the
variability in the measure is due to local energy efficiency and hardware choice.